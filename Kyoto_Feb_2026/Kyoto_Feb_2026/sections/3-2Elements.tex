\section{Elements Formation and Distribution}

%------------------------------------------------
\begin{frame}{Elements distribution's dependence on $\theta$}
\begin{columns}

\column{0.6\textwidth}
\centering
% --- First row ---
\includegraphics[width=0.45\linewidth]{../fig/blh_vlr_hr_constatm/Masses_sections_theta.pdf}%
\hspace{0.04\linewidth}%
\includegraphics[width=0.45\linewidth]{../fig/sfho_vlr_hr_constatm/Masses_sections_theta.pdf}

\vspace{2mm}

% --- Second row ---
\includegraphics[width=0.45\linewidth]{../fig/dd2_135_vlr_hr_constatm/Masses_sections_theta.pdf}%
\hspace{0.04\linewidth}%
\includegraphics[width=0.45\linewidth]{../fig/dd2_180_vlr_hr_constatm/Masses_sections_theta.pdf}

\column{0.4\textwidth}
\scriptsize
\begin{itemize}
	\item Bulk of heavy elements for $\theta \lesssim 36^\circ$
	\item Largest mass of heavy $r$-process elements on top of remnant for DD2\_q1.67
	\item SFHo\_q1.0 has the smallest production of H and He
\end{itemize}

\end{columns}
\end{frame}

%------------------------------------------------
\begin{frame}{Elements distribution's dependence on $\phi$}
\begin{columns}

\column{0.6\textwidth}
\centering
% --- First row ---
\includegraphics[width=0.45\linewidth]{../fig/blh_vlr_hr_constatm/Masses_sections_phi.pdf}%
\hspace{0.04\linewidth}%
\includegraphics[width=0.45\linewidth]{../fig/sfho_vlr_hr_constatm/Masses_sections_phi.pdf}

\vspace{2mm}

% --- Second row ---
\includegraphics[width=0.45\linewidth]{../fig/dd2_135_vlr_hr_constatm/Masses_sections_phi.pdf}%
\hspace{0.04\linewidth}%
\includegraphics[width=0.45\linewidth]{../fig/dd2_180_vlr_hr_constatm/Masses_sections_phi.pdf}

\column{0.4\textwidth}
\scriptsize
\begin{itemize}
	\item SFHo\_q1.0 and DD2\_q1.0 show stronger $\phi$ dependence for elements such as ${}^{134}_{54}\mathrm{Xe}$
	\item SFHo\_q1.0 and DD2\_q1.0 production varies by less than a factor $\lesssim 2$
	\item Largest variations for BLh\_q1.43 and DD2\_q1.67 occur for $A \gtrsim 134$
	\item Variations can reach factors of $\sim 5$--$10$
\end{itemize}

\end{columns}
\end{frame}

%------------------------------------------------
\begin{frame}{Sky Maps}
\begin{itemize}
    \item Sky
    \item Maps
\end{itemize}
\end{frame}

%------------------------------------------------
\begin{frame}{Temporal Dependence}
\begin{columns}

\column{0.6\textwidth}
\centering
% --- First row ---
\includegraphics[width=0.45\linewidth]{../fig/blh_vlr_hr_constatm/BLH_polar.pdf}%
\hspace{0.04\linewidth}%
\includegraphics[width=0.45\linewidth]{../fig/sfho_vlr_hr_constatm/SFHO_polar.pdf}

\vspace{2mm}

% --- Second row ---
\includegraphics[width=0.45\linewidth]{../fig/dd2_135_vlr_hr_constatm/DD2S_polar.pdf}%
\hspace{0.04\linewidth}%
\includegraphics[width=0.45\linewidth]{../fig/dd2_180_vlr_hr_constatm/DD2A_polar.pdf}

\column{0.4\textwidth}
\scriptsize
\begin{itemize}
	\item WinNet fed with datasets evolved for 300\,ms and 1\,s
	\item Symmetric binaries show limited sensitivity
	\item BLh\_q1.43 polar regions affected near the first $r$-process peak
	\item DD2\_q1.67 polar regions affected for $A \gtrsim 130$
	\item Lanthanide and actinide expansion (?)
\end{itemize}

\end{columns}
\end{frame}

%------------------------------------------------
\begin{frame}{Temporal Dependence}
\centering
\includegraphics[width=0.9\linewidth]{../fig/dd2_180_vlr_hr_constatm/Sky_locations_compare.pdf}
\end{frame}

%------------------------------------------------
\begin{frame}{${}^{56}_{28}\mathrm{Ni}$ impact}
\begin{columns}[T]

\column{0.7\textwidth}
\centering
% --- First row ---
\includegraphics[width=0.45\linewidth]{../fig/blh_vlr_hr_constatm/Energy_evolution_zoom_BLH_q1.43.pdf}%
\hspace{0.04\linewidth}%
\includegraphics[width=0.45\linewidth]{../fig/sfho_vlr_hr_constatm/Energy_evolution_zoom_SFHO_q1.0.pdf}

\vspace{2mm}

% --- Second row ---
\includegraphics[width=0.45\linewidth]{../fig/dd2_135_vlr_hr_constatm/Energy_evolution_zoom_DD2_q1.0.pdf}%
\hspace{0.04\linewidth}%
\includegraphics[width=0.45\linewidth]{../fig/dd2_180_vlr_hr_constatm/Energy_evolution_zoom_DD2_q1.67.pdf}

\column{0.3\textwidth}

\begin{tcolorbox}[
    colback=white,
    colframe=blue,
    colbacktitle=blue!50!black!50,
    title=Ejecta masses,
    fonttitle=\bfseries,
    rounded corners,
    boxrule=1pt
]
\tiny
\begin{tabular}{c|c}
Model & $M_{\mathrm{ej}}\, [M_\odot]$ \\
\hline
BLh\_q1.43 & $8.37\times10^{-3}$ \\
SFHo\_q1.0 & $4.58\times10^{-3}$ \\
DD2\_q1.0  & $4.63\times10^{-3}$ \\
DD2\_q1.67 & $1.38\times10^{-2}$
\end{tabular}
\end{tcolorbox}

\vspace{2mm}

\begin{tcolorbox}[
    colback=white,
    colframe=blue,
    colbacktitle=blue!50!black!50,
    title=Nickel masses,
    fonttitle=\bfseries,
    rounded corners,
    boxrule=1pt
]
\tiny
\begin{tabular}{c|c}
Model & $M_{{}^{56}\mathrm{Ni}}\, [M_\odot]$ \\
\hline
BLh\_q1.43 & $4.60\times10^{-4}$ \\
SFHo\_q1.0 & $1.10\times10^{-6}$ \\
DD2\_q1.0  & $1.59\times10^{-4}$ \\
DD2\_q1.67 & $3.10\times10^{-4}$
\end{tabular}
\end{tcolorbox}

\end{columns}
\end{frame}
